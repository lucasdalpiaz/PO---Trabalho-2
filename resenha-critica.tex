\documentclass[a4paper,oneside]{article}


\pagestyle{myheadings}

%%%%%%%%%%%%%%%%%%%%%%%%%%%
% Pacotes para acentua��o %
%%%%%%%%%%%%%%%%%%%%%%%%%%%
%\usepackage{cite}
%\usepackage{natbib}     %% note the absence of options
% Change how references are included (see the natbib package).
%\setcitestyle{square,numbers,authoryear}
%\setcitestyle{authoryear}
\usepackage[alf]{abntex2cite}
\usepackage[utf8]{inputenc}
\usepackage[T1]{fontenc}
\usepackage{ae}
\usepackage{booktabs}
%\usepackage[ansinew]{inputenc}
\usepackage{graphicx}

\usepackage[brazilian]{babel}

%%%%%%%%%%%%%%%%%%%%%%%%%%%%%%%%%%%%%%%%%%%%%%%%%%%
\usepackage{algpseudocode,algorithm}
% Declaracoes em Português
\algrenewcommand\algorithmicend{\textbf{fim}}
\algrenewcommand\algorithmicdo{\textbf{faça}}
\algrenewcommand\algorithmicwhile{\textbf{enquanto}}
\algrenewcommand\algorithmicfor{\textbf{para}}
\algrenewcommand\algorithmicif{\textbf{se}}
\algrenewcommand\algorithmicthen{\textbf{então}}
\algrenewcommand\algorithmicelse{\textbf{senão}}
\algrenewcommand\algorithmicreturn{\textbf{devolve}}
\algrenewcommand\algorithmicfunction{\textbf{função}}

% Rearranja os finais de cada estrutura
\algrenewtext{EndWhile}{\algorithmicend\ \algorithmicwhile}
\algrenewtext{EndFor}{\algorithmicend\ \algorithmicfor}
\algrenewtext{EndIf}{\algorithmicend\ \algorithmicif}
\algrenewtext{EndFunction}{\algorithmicend\ \algorithmicfunction}

% O comando For, a seguir, retorna 'para #1 -- #2 até #3 faça'
\algnewcommand\algorithmicto{\textbf{até}}
\algrenewtext{For}[3]%
{\algorithmicfor\ #1 $\gets$ #2 \algorithmicto\ #3 \algorithmicdo}

%%%%%%%%%%%%%%%%%%%%%%%%%%%%%%%%%%%%%%%%%%%%%%%%%%%


\usepackage{nomencl}
\makenomenclature
\renewcommand{\nomname}{Lista de Símbolos}


\linespread{1.5} % espa�amento entre linhas



% horizontal
\setlength{\hoffset}{-1in}

\setlength{\oddsidemargin}{3.0cm} 

\setlength{\textwidth}{160mm} % (210mm - 30mm - 20mm)

\setlength{\parindent}{1.25cm} % identa��o de cada par�grafo

% vertical
\setlength{\voffset}{-1in}
\addtolength{\voffset}{2.0cm}

\setlength{\topmargin}{0.0cm}

\setlength{\headheight}{5mm}
\setlength{\headsep}{5mm}

\setlength{\textheight}{247mm} % (297mm - 30mm - 20mm)



\title{Atividade 1}

\author{\\
\\

Universidade Federal Fluminense\\



} 


\begin{document}


\pagenumbering{arabic}

\begin{titlepage}
  \begin{center}
\Large{\textsc{Universidade Federal Fluminense} \\
  %         \textsc{Pólo Universitário de Rio das Ostras} \\ 
           \textsc{Mestrado de Engenharia de Produção e Sistemas Computacionais}
          }
    \par\vfill
    \LARGE{Lucas Campos Dal Piaz de Souza\\Arthur Neves Fraga Serejo}
    \par\vfill
%    \bigskip
    \LARGE{Modelagem de Sistemas de Software}\\
    \LARGE{Trabalho Final}
    \par\vfill
    \Large{Rio das Ostras-RJ\\Agosto, 2020.}
  \end{center}
\end{titlepage}
\newpage
\tableofcontents
\newpage
%\citeonline{nome_referenciado no refs}  \cite{nome_referenciado no refs}


%\thispagestyle{empty} 



%PENSEI EM UM PROBLEMA DAQUELES DO TIPO MIX DE PRODUÇÃO AO INVÉS DE UM PROBLEMA DE TRANSPORTE, ACHO QUE FEZ MAIS SENTIDO.


\section {Estudo de caso I}
\subsection{Questão 1}
$x_{ij}$: Quantidade a ser produzida do conjunto i.$\forall$ i $\in$ \{1 - Alfa, 2 - Beta, 3 - Gama\}. $\forall$ j $\in$  \{1 - Corte, 2 - Preparação, 3 - Montagem, 4 - Pintura, 5 - Embalagem, 6 - Pintura\}.\\
$R_j$: Custos unitários dos recursos j.
\\
$C_i$: Contribuições Marginais dos Conjuntos i.
\\
$P_{ij}$: Matriz de utilização unitária de recursos j para fabricação dos conjuntos i.
\\
$A_i$: Alocação Inicial de recurso por seção.
\\
O objetivo maximizar a produção mensal:
\begin{equation}
  MAX Z =  \sum_{i} \sum_{j}  (C_{i}x_{ij}) - \sum_{j} \sum_{i}  (R_{i}x_{ij})
\end{equation}
S.a.:
\begin{equation}
    \sum_{j}(\sum_{i} ( P_{ij}x_{i}) \leq  A_{j} )
\end{equation}
%\begin{equation}
%     \sum_{j} (\sum_{i}x_{i} \geq R_{j})
%\end{equation}

\subsection{Questão 2}
\subsection{Questão 3}
\subsection{Questão 4}

\begin{equation}
    \sum_{i} ( \sum_{j}  44(P_{ij}x_{ij}) \leq  A_i)
\end{equation}
\subsection{Questão 3}
\section {Estudo de caso II}

\subsection{Caso 1}
Estudo de caso para o mercado atual:\\
$x_{ij}$: Quantidade a ser transportada da origem i para o destino j. $\forall$ i $\in$ \{1 - Porto Alegre, 2 - Belo Horizonte, 3 - Salvador\}. $\forall$ j $\in$  \{1 - Curitiba, 2 - Recife, 3 - São Paulo, 4 - Rio de Janeiro, 5 - Brasília\}.
\\
$D_j$: Demanda dos destinos j.
\\
$C_{ij}$: Custos unitários de transporte.
\\
$O_i$: Disponibilidade das origens i.
\\
O objetivo minimizar os custos de transporte entre as origens e destino:
\begin{equation}
  MIN Z =  \sum_{i}\sum_{j} (C_{ij}x_{ij})
\end{equation}
S.a.:
\begin{equation}
    \sum_{i}(\sum_{j} x_{ij}) \leq  O_{i} )
\end{equation}
\begin{equation}
      \sum_{j} (\sum_{i}x_{ij} \geq D_{j})
\end{equation}
O custo total é de 107800.
\subsection{Caso 2}
Estudo de caso para o mercado futuro com aumento de demanda:
\begin{itemize}
\item{Opção A}\\
O custo total é de 173200.
\item{Opção B}\\
O custo total é de 162200.
\end{itemize}
A melhor opção para planejamento do futuro da empresa é a B, pois o custo é menor.
\end{document}